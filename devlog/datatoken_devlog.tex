%%%%%%%%%%%%%%%%%%%%%%%%%%%%%%%%%%%%%%%%%
% University Assignment Title Page 
% LaTeX Template
% Version 1.0 (27/12/12)
%
% This template has been downloaded from:
% http://www.LaTeXTemplates.com
%
% Original author:
% WikiBooks (http://en.wikibooks.org/wiki/LaTeX/Title_Creation)
%
% License:
% CC BY-NC-SA 3.0 (http://creativecommons.org/licenses/by-nc-sa/3.0/)
% 
% Instructions for using this template:
% This title page is capable of being compiled as is. This is not useful for 
% including it in another document. To do this, you have two options: 
%
% 1) Copy/paste everything between \begin{document} and \end{document} 
% starting at \begin{titlepage} and paste this into another LaTeX file where you 
% want your title page.
% OR
% 2) Remove everything outside the \begin{titlepage} and \end{titlepage} and 
% move this file to the same directory as the LaTeX file you wish to add it to. 
% Then add \input{./title_page_1.tex} to your LaTeX file where you want your
% title page.
%
%%%%%%%%%%%%%%%%%%%%%%%%%%%%%%%%%%%%%%%%%
%\title{Title page with logo}
%----------------------------------------------------------------------------------------
%	PACKAGES AND OTHER DOCUMENT CONFIGURATIONS
%----------------------------------------------------------------------------------------

\documentclass[12pt]{article}
\usepackage[english]{babel}
\usepackage[utf8x]{inputenc}
\usepackage{amsmath}
\usepackage{graphicx}
\usepackage{epstopdf}
\usepackage[colorinlistoftodos]{todonotes}
\usepackage{listings}
\usepackage{hyperref}
%\usepackage{harvard}

\begin{document}

\begin{titlepage}

\newcommand{\HRule}{\rule{\linewidth}{0.5mm}} % Defines a new command for the horizontal lines, change thickness here

\center % Center everything on the page
 
%----------------------------------------------------------------------------------------
%	HEADING SECTIONS
%----------------------------------------------------------------------------------------

\textsc{\LARGE XJTLU}\\[1.5cm] % Name of your university/college
\textsc{\Large DataToken Developer Diary}\\[0.5cm] % Major heading such as course name
\textsc{\large Solidity Smart Contract}\\[0.5cm] % Minor heading such as course title

%----------------------------------------------------------------------------------------
%	TITLE SECTION
%----------------------------------------------------------------------------------------

\HRule \\[0.4cm]
{ \huge \bfseries P2P Cellular Data Sharing}\\[0.4cm] % Title of your document
\HRule \\[1.5cm]
 
%----------------------------------------------------------------------------------------
%	AUTHOR SECTION
%----------------------------------------------------------------------------------------

\begin{minipage}{0.4\textwidth}
\begin{flushleft} \large
\emph{Author:}\\
Zhuoqun \textsc{Liu} % Your name
\end{flushleft}
\end{minipage}
~
\begin{minipage}{0.4\textwidth}
\begin{flushright} \large
\emph{Supervisor:} \\
Dr. Siyi \textsc{Wang} % Supervisor's Name
\end{flushright}
\end{minipage}\\[2cm]

% If you don't want a supervisor, uncomment the two lines below and remove the section above
%\Large \emph{Author:}\\
%John \textsc{Smith}\\[3cm] % Your name

%----------------------------------------------------------------------------------------
%	DATE SECTION
%----------------------------------------------------------------------------------------

{\large \today}\\[1.5cm] % Date, change the \today to a set date if you want to be precise

%----------------------------------------------------------------------------------------
%	LOGO SECTION
%----------------------------------------------------------------------------------------

\includegraphics[width=0.33\linewidth]{XJTLU_shield.eps}\\[0.3cm] % Include a department/university logo - this will require the graphicx package
 
%----------------------------------------------------------------------------------------

\vfill % Fill the rest of the page with whitespace

\end{titlepage}


\begin{abstract}
This is a placeholder line for Abstract.
\end{abstract}

\section{Introduction}
DataToken is the name of the smart contract to be developed as my Final Year Project (FYP) in XJTLU.
All DataToken projects will be named with postfix``alpha'' since the propotype development is far from finished yet.
Versions are depicted in *.* format e.g. 0.0; increment on the number before the dot means new functions or features are added, increment on the number after the dot means minor changes are added to source code.
\section{Diary}
\label{sec:examples}
\subsection{Parameters Settings}
\begin{itemize}
    \item User 
    \begin{itemize}
        \item user index
        \item userId
    \end{itemize}
    
\end{itemize}

\subsection{DataToken Alpha 0.0}
version alpha 0.0 
initialy designed functions
createAccount, askforSharing

\subsection{DataToken Alpha 0.1 20171014 Fri}
The contract should be an ether pool (etherbase)
thus it should allow user charge their account by putting ether in to the contranct.
the first method to put ether in is the provide value when creating a user account.
A second way is to call topUp function to send ether to this contract and get some token.
By design, the token distributed by the contract should be at a constant exchange rate to ether.
currently, the rate is set to be 1 token = 1 wei which is the smallest ether unit.

A withdraw function is created corresponding to topUp function.
This function should allow user account to exchange their token back to ether in their ether address.
But the chellange is, when sending ether from the contract, the gas fee is to be paid by the sender, if the sender here is the message sender i.e. the user,
the user will pay for the gas, however, if the sender is considered as the contract itself, there will be a problem.
If the contract is charged gas for each transaction (the contract will loss at least 0.001 ether for each transaction according to current gas price),
the contract etherbase will fail due to too many withdraw transactions. The contract itself is not making any profit but will have to pay some fee due to users' transaction,
that's not fair.
Then an experiment should be held to examine how the ethereum network perform such a transaction from the contract with msg.sender a user to be the caller of the function.
$0.001$ ether = $10^15$  wei which is a large amount of loss in terms of wei.
    experiment design:
    create the contract with external address A;
    create user account with external address B, charge 0.01 ether for token;
        expecting: the contract has 0.01 ether and address B pay 0.01 ether plus gas fee;
    address B call withdraw function to withdraw 0.01 ether;
        expecting: the contract send 0.01 ether to B and B pay the gas, resulting B receive 0.01 minus gas fee.
    An easy way to examine the behavior of the function (actually the contract convention) is to set a getter for the contract.
    top-up the contract first, suppose the contract possesses \_amountA wei
        check the contract balance;
    withdraw \_amountB wei by calling function withdraw as external address (user)
        check the contract balance;
        if the balance == \_amountA - \_amountB
            the transaction fee was paid by the msg.sender i.e. the user account.
        else
            the fee was paid by the contract

    If by convention the user will pay for calling a function that send ether from contract etherbase, 
    the withdraw function need not to have a mechanism to make the ether in contract intact
    
The contract shouldn't be paying that fee for transfering back. 



\subsection{Diary 20171016 Mon}
Mapping a host address to a guest address will make the link unique, thereby the sevice can only be recorded one on one which is no good for practical use.
I'm now searching for a datastructurea like array which can be marked as related to one host address so that all guests of the same host address can be stored in it.
In the documentation of solidity \^0.4.18, mapping types expression $$mapping(KeyType => ValueType)$$
allows KeyType to be almost any type except for mapping type; ValueType to any type including mapping type.
This mapping feature can merge mappings from guest to payment status.
And here is a solution for linking host to guest.\href{https://ethereum.stackexchange.com/questions/27053/how-to-create-a-mapping-of-string-and-struct-array-in-solidity}{The following code is a good demonstration from stackexchange.}
\begin{lstlisting}
    pragma solidity ^0.4.11;
    
    contract AuthorizationManager{
        struct User{
          string userId;
          uint roleId;
        }
    
        mapping (string => User[]) companyUserMap;
    
        function addUser(string _key,string _userId, uint _roleId){
            companyUserMap[_key].push(User(_userId,_roleId));
        }
    
        function removeSingleUser(string _key){
            companyUserMap[_key].length--;
        }
    }
\end{lstlisting}
\subsection{Diary 20171017 Tue}
How should a ledger be like? An individual ledger or a public ledger to record everyone?
\subsection{Mathematics}

\LaTeX{} is great at typesetting mathematics. Let $X_1, X_2, \ldots, X_n$ be a sequence of independent and identically distributed random variables with $\text{E}[X_i] = \mu$ and $\text{Var}[X_i] = \sigma^2 < \infty$, and let
$$S_n = \frac{X_1 + X_2 + \cdots + X_n}{n}
      = \frac{1}{n}\sum_{i}^{n} X_i$$
denote their mean. Then as $n$ approaches infinity, the random variables $\sqrt{n}(S_n - \mu)$ converge in distribution to a normal $\mathcal{N}(0, \sigma^2)$.

\subsection{Lists}

You can make lists with automatic numbering \dots

\begin{enumerate}
\item Like this,
\item and like this.
\end{enumerate}
\dots or bullet points \dots
\begin{itemize}
\item Like this,
\item and like this.
\end{itemize}
We hope you find write\LaTeX\ useful, and please let us know if you have any feedback using the help menu above.

%\bibliographystyle{agsm}
\bibliographystyle{IEEEtran}
\bibliography{reference}
\end{document}