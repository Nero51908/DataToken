%% Generated by Sphinx.
\def\sphinxdocclass{report}
\documentclass[letterpaper,10pt,english]{sphinxmanual}
\ifdefined\pdfpxdimen
   \let\sphinxpxdimen\pdfpxdimen\else\newdimen\sphinxpxdimen
\fi \sphinxpxdimen=.75bp\relax

\usepackage[utf8]{inputenc}
\ifdefined\DeclareUnicodeCharacter
 \ifdefined\DeclareUnicodeCharacterAsOptional
  \DeclareUnicodeCharacter{"00A0}{\nobreakspace}
  \DeclareUnicodeCharacter{"2500}{\sphinxunichar{2500}}
  \DeclareUnicodeCharacter{"2502}{\sphinxunichar{2502}}
  \DeclareUnicodeCharacter{"2514}{\sphinxunichar{2514}}
  \DeclareUnicodeCharacter{"251C}{\sphinxunichar{251C}}
  \DeclareUnicodeCharacter{"2572}{\textbackslash}
 \else
  \DeclareUnicodeCharacter{00A0}{\nobreakspace}
  \DeclareUnicodeCharacter{2500}{\sphinxunichar{2500}}
  \DeclareUnicodeCharacter{2502}{\sphinxunichar{2502}}
  \DeclareUnicodeCharacter{2514}{\sphinxunichar{2514}}
  \DeclareUnicodeCharacter{251C}{\sphinxunichar{251C}}
  \DeclareUnicodeCharacter{2572}{\textbackslash}
 \fi
\fi
\usepackage{cmap}
\usepackage[T1]{fontenc}
\usepackage{amsmath,amssymb,amstext}
\usepackage{babel}
\usepackage{times}
\usepackage[Bjarne]{fncychap}
\usepackage{sphinx}

\usepackage{geometry}

% Include hyperref last.
\usepackage{hyperref}
% Fix anchor placement for figures with captions.
\usepackage{hypcap}% it must be loaded after hyperref.
% Set up styles of URL: it should be placed after hyperref.
\urlstyle{same}
\addto\captionsenglish{\renewcommand{\contentsname}{Contents:}}

\addto\captionsenglish{\renewcommand{\figurename}{Fig.}}
\addto\captionsenglish{\renewcommand{\tablename}{Table}}
\addto\captionsenglish{\renewcommand{\literalblockname}{Listing}}

\addto\captionsenglish{\renewcommand{\literalblockcontinuedname}{continued from previous page}}
\addto\captionsenglish{\renewcommand{\literalblockcontinuesname}{continues on next page}}

\addto\extrasenglish{\def\pageautorefname{page}}

\setcounter{tocdepth}{1}



\title{DataTokenAlpha Documentation}
\date{Mar 06, 2018}
\release{}
\author{Zhuoqun Liu}
\newcommand{\sphinxlogo}{\vbox{}}
\renewcommand{\releasename}{}
\makeindex

\begin{document}

\maketitle
\sphinxtableofcontents
\phantomsection\label{\detokenize{index::doc}}


This is the documentation of DataTokenAlpha.
DatatokenAlpha is my final year project for undergraduate.


\chapter{Indices}
\label{\detokenize{index:indices}}\label{\detokenize{index:welcome-to-the-documentation-of-datatokenalpha}}\begin{itemize}
\item {} 
\DUrole{xref,std,std-ref}{genindex}

\item {} 
\DUrole{xref,std,std-ref}{modindex}

\item {} 
\DUrole{xref,std,std-ref}{search}

\end{itemize}


\section{Introduction}
\label{\detokenize{index:introduction}}
This project aims to develop a scheme to allow peer-to-peer (P2P) cellular data sharing and to
construct a prototype Ethereum smart contract with respect to the scheme using Solidity which is
a blockchain oriented programming language. Smart contract DataToken-Alpha (latest version 0.2.2)
as the scheme implementation has been under development. Hence, DataToken, a token of Ether
(Intrinsic Cryptocurrency of Ethereum) is created for data service transfer recording.The uniqueness
of such implementation is that all the information held by the contract is protected by Ethereum
network based on blockchain technology from computational hacking.

The implementation is started from analyzing behaviors that a ledger system is expected to have.
By design, the contract is firstly able to offer and manage token like a bank in real world
and secondly, it is capable to work as a cellular data service trade centre where users of such
contract are able to transfer Internet access in terms of cellular data usage and token.
Solidity code thud is used to implement necessary features with respect to the behaviors mentioned above.
Finally, a prototype of DataToken-Alpha for autonomous data trading is ready for further polish
in the next semester when this project will focus on upgrading DataToken-Alpha to be a practically
deployable smart contract as close as possible.


\section{Contract Variables}
\label{\detokenize{index:contract-variables}}
Contracts variables are declared at the beginning of solidity contract body.
In this section, variable types and availability are indicated by source code definition
below each mentioned variables.


\subsection{owner}
\label{\detokenize{index:owner}}
\fvset{hllines={, ,}}%
\begin{sphinxVerbatim}[commandchars=\\\{\}]
\PYG{n}{address} \PYG{n}{owner}\PYG{p}{;}
\end{sphinxVerbatim}


\subsection{tokenName}
\label{\detokenize{index:tokenname}}

\section{Internal Functions}
\label{\detokenize{index:internal-functions}}

\section{Public Functions}
\label{\detokenize{index:public-functions}}


\renewcommand{\indexname}{Index}
\printindex
\end{document}